\documentclass{beamer}
\usepackage[T1]{fontenc}
\usepackage[utf8]{inputenc}

\usetheme{Berkeley}
\setbeamercolor{structure}{fg=red!30!}

\setbeamercovered{dynamic}
\setbeamertemplate{sidebar  canvas  left}[vertical shading][top=pink,bottom=white]

\logo{%
	  \includegraphics[height=.2\paperheight]{sigillo}~%
		  }%

%\usebackgroundtemplate{
%     \parbox{\paperwidth}{\includegraphics[height=2\paperheight]{modulor}}} 

\setbeamertemplate{navigation symbols}
{
 \usebeamerfont{date in head/foot}\pgfsetfillopacity{1}\insertshortdate{}\hspace*{5em}
      
  \hspace{1em}
  \includegraphics[width=3cm,higth=2cm]{images} 
  \hspace{1em}  

  \insertframenumber/\inserttotalframenumber
    }
   
\setcounter{page}{1} 
\pagenumbering{arabic}

\begin{document}

\begin{frame}

\frametitle{Presentazione}
\framesubtitle{\textit{di Rosamaria Donnici}}

\begin{columns}
\begin{column}{0.35\textwidth}

   {\textbf{\itshape Villa Sovoye} } \\  1929 - 1931 \\ 
   {\itshape "Les Heures Claires"} \\  di Pierre Jeanneret e Charles-Edouard   Jeanneret \\ detto \emph{Le Corbusier}  \\ 
\end{column}

\begin{column}{0.7\textwidth}
\begin{itemize} 
  \item<1-| alert@1> \emph{evoluzione} : promenade architecturale
  \item<2-| alert@2> \emph{metrica} :  il Modulor
  \item<3-| alert@3> \emph{costrutto} : la sezione aurea
  \item<4-| alert@4> \emph{tema centrale} : il cemento
\end{itemize}

\end{column}
\end{columns}
  
\end{frame}

\setbeamercolor{structure}{fg=yellow!60!}
\setbeamercolor{frametitle}{fg=black}
\setbeamercolor{navigation symbols}{fg=black!80!, bg=green}
\setbeamertemplate{sidebar  canvas  left}[vertical shading][top=yellow,bottom=white]
 
\usebackgroundtemplate{
\parbox{\paperwidth}{\includegraphics[height=1\paperheight]{POISSY}}}

\begin{frame}
\frametitle{Villa Savoye}
\framesubtitle{\textit{Il luogo}}
\end{frame}

\usebackgroundtemplate{}
\begin{frame}

\frametitle{Villa Savoye}
\framesubtitle{\textit{L'incubo di Madame Savoye}}
\setbeamercolor{title}{black}

\begin{columns}
\begin{column}{0.5\textwidth}
  progettata per essere una semplice casa di campagna \\nei pressi di Parigi
 \\nel 1965 \\- Le Corbusier era ancora in vita - è stata inserita nella lista dei monumenti storici - architettonici francesi\\ candidata tra i patrimoni culturali UNESCO \\
\textbf{è considerata manifesto dell'architettura moderna}
\end{column}

\begin{column}{0.4\textwidth}
\includegraphics[scale=0.8]{220px}

\end{column}
\end{columns}

\end{frame}

\usebackgroundtemplate{
\parbox{\paperwidth}{\includegraphics[height=1\paperheight]{POISSY}}}

\begin{frame}

\frametitle{Villa Savoye}
\framesubtitle{\textit{Il risultato}}
\setbeamercolor{title}{black}

\includegraphics[height=0.5\paperheight]{index}
\end{frame} 

 
 \usebackgroundtemplate{}
 \setbeamercolor{structure}{fg=red!30!}
 \setbeamertemplate{sidebar  canvas  left}[vertical shading] [top=pink,bottom=white]
 
\begin{frame} 
\frametitle{Villa Savoye}
\framesubtitle{\textit{Concetti base}}
\setbeamercolor{title}{fg=black!80!black,bg=black!20!black}

\begin{itemize}   
   \item<1-| alert@1> \emph{pilotis} : la "foret" conferisce un effetto di galleggiamento sostenendo comunque la struttura 
   \item<2-| alert@2> \emph{plan libre} : la disposizione delle pareti è senza schemi stabiliti
   \item<3-| alert@3> \emph{façade libre} : sono possibili dislocazioni di bucature e oggetti sulla facciata
   \item<4-| alert@4> \emph{fenêtre en longueur} : le finestre tagliano la facciata in tutta la lunghezza
   \item<5-| alert@5> \emph{toit-terasse} : il tetto diventa un possibile spazio verde alternativo  

  \end{itemize}
  
\end{frame}

\usebackgroundtemplate{}

\begin{frame}

\frametitle{Villa Savoye}
\framesubtitle{\textit{I pilastri}}

\begin{columns}
\begin{column}{0.6\textwidth}

Sostituiscono i voluminosi setti in muratura con sostegni molto esili poggiati su plinti in calcestruzzo che penetrano fin detto il terreno. Su essi si sviluppano i solai in lastre piane di calcestruzzo armato. L'edificio, in questo modo è retto da alti piloni puntiformi che separano la costruzione dal terreno e dall'umidità. 
 \end{column}
 
\begin{column}{0.4\textwidth}
  \includegraphics[height=0.5\paperheight]{Seperation}
   
 \end{column}
 \end{columns}

 \end{frame}

 \begin{frame}

\frametitle{Villa Savoye}
\framesubtitle{\textit{I pilastri}}

L'area ora disponibile viene utilizzata come garage consentendo la circolazione delle automobili sin sotto gli edifici con tre parcheggi disponibili
 
  \includegraphics[height=0.45\paperheight]{pilastri}

 \end{frame}
 
\begin{frame} 

\frametitle{Villa Savoye}
\framesubtitle{\textit{I pilastri}}

\begin{columns}
\begin{column}{0.4\textwidth}
 
\includegraphics[height=0.6\paperheight]{pianoterra} 

 \end{column}
 
\begin{column}{0.5\textwidth}

Il passo è di 4,75m su maglia non perfettamente quadrata con una ripetizione di 5 pilastri(moduli) - non tutti in linea e in falso - di 25cm di diametro e 2,64m di altezza

\end{column}
\end{columns}
  
\end{frame}


\begin{frame}
\frametitle{Villa Savoye}
\framesubtitle{\textit{Pianta libera}}

\includegraphics[height=0.5\paperheight]{0180}

\end{frame}

\begin{frame}

\frametitle{Villa Savoye}
\framesubtitle{\textit{Pianta libera}}

~~~~\includegraphics[height=0.3\paperheight]{plan}

Grazie al cemento armato e ai pilastri che sostengono le solette, i  piani non si sovrappongono con tramezzature ma sono indipendenti. La casa è priva di muri portanti e divisori e ciò permette di ottenere grandi spazi interni con libertà di planimetria 

\end{frame}

\begin{frame}
\frametitle{Villa Savoye}
\framesubtitle{\textit{Una vista dell'interno - La cucina}}

\includegraphics[height=0.5\paperheight]{lart2}

\end{frame}


\begin{frame}
\frametitle{Ville Savoye}
\framesubtitle{\textit{Una vista dell'interno - Il bagno}}

\includegraphics[height=0.55\paperheight]{bagno}
\end{frame}

\usebackgroundtemplate{}

\begin{frame}
\frametitle{Villa Savoye}
\framesubtitle{\textit{Facciata libera}}

~~~~\includegraphics[height=0.2\paperheight]{b69}

\begin{columns}
\begin{column}{0.6\textwidth}

La facciata della \emph{"machine à habiter"} non sono più costituite di murature strutturali, ma semplicemente da una serie di elementi orizzontali e verticali i cui vuoti possono essere tamponati a piacimento, sia con pareti isolanti che con infissi trasparenti e scorrevoli che ampliano il concetto di spazio unico
 
\end{column}

\begin{column}{0.4\textwidth} 
\includegraphics[height=0.2\paperheight]{libera}

\end{column}
\end{columns}

\end{frame}

\begin{frame}
\frametitle{Villa Savoye}
\framesubtitle{\textit{Il giardino pensile}}

\includegraphics[height=0.5\paperheight]{22}

\end{frame}

\begin{frame}
\frametitle{Villa Savoye}
\framesubtitle{\textit{Finestre a nastro}}

\begin{columns}
\begin{column}{0.4\textwidth}

Diretta conseguenza della modularità, il progetto delle aperture prevede una corrispondenza tra campata e finestra.I maggiori spazi interni sono composti da combinazioni di singole unità con finestre scorrevoli in legno di 1x2,5 m ad anta fissa \\

\end{column}

\begin{column}{0.5\textwidth}

\includegraphics[height=0.25\paperheight]{86a}

\end{column}
\end{columns}

\end{frame}

\usebackgroundtemplate{
\parbox{\paperwidth}{\includegraphics[height=1\paperheight]{13}}}

\begin{frame}
\frametitle{Villa Savoye}
\framesubtitle{\textit{Il tetto-giardino}}

\begin{columns}
\begin{column}{0.5\textwidth}

\textbf{Il terreno tra i giunti delle lastre di copertura in calcestruzzo armato permette di seminare erba e piante che hanno una funzione coibente per i piani inferiori}

\end{column}

\begin{column}{0.4\textwidth}
\end{column}
\end{columns}

\end{frame}

\usebackgroundtemplate{}

\begin{frame}
\frametitle{Villa Savoye}
\framesubtitle{\textit{La promenade}}

\includegraphics[height=0.5\paperheight]{prom}

\end{frame}

\begin{frame}
\frametitle{Villa Savoye}
\framesubtitle{\textit{La promenade}}


\begin{columns}
\begin{column}{0.5\textwidth}

Un unico percorso senza alcuna barriera architettonica parte dal garage - motore e idea del luogo abitativo - lungo il prisma monocolore spezzato dai vuoti delle finestre del primo piano - parte viva della casa - e lungo la rampa esterna del giardino pensile fino a sbarcare - come sul ponte di una nave - nel solarium  protetto da una parete tagliavento che riprende la forma delle curve al piano terra

\end{column}

\begin{column}{0.5\textwidth}

\includegraphics[height=0.4\paperheight]{d98}

\end{column}
\end{columns}
\end{frame}

\begin{frame}

\frametitle{Villa Savoye}
\framesubtitle{\textit{Piantine}}

\includegraphics[height=0.75\paperheight]{piante}

\end{frame}

\begin{frame}

\frametitle{Villa Savoye}
\framesubtitle{\textit{Piantine}}

\includegraphics[height=0.85\paperheight]{piantine}

\end{frame}

\begin{frame}

\frametitle{Villa Savoye}
\framesubtitle{\textit{Piantine}}

\includegraphics[height=0.75\paperheight]{Villa}

\end{frame}

\begin{frame}     

\frametitle{Conclusione}
\framesubtitle{\textit{Un po' di numeri}}

\begin{columns}
\begin{column}{0.4\textwidth}

Il Modulor è una scala metrica basata su una progressione dimensionale delle
diverse parti del corpo umano tale per cui una misura era sempre la sezione aurea della successiva che, secondo Le Corbusier, soddisfa a piena
le esigenze ergonomiche ed estetiche dell'architettura.

\end{column}
\begin{column}{0.3\textwidth}

\includegraphics[scale=0.3]{modd}

\end{column}
\end{columns}

\end{frame}

\begin{frame}     

\frametitle{Conclusione}
\framesubtitle{\textit{Un po' di numeri}}

\begin{columns}
\begin{column}{0.5\textwidth}

La sezione aurea o \emph{costante di Fidia} o \emph{proporzione divina} indica il rapporto fra due lunghezze disuguali, delle quali la maggiore è medio proporzionale tra la minore e la somma delle due: \\ AB : AC = AC : CB = 1,618 

\end{column}
\begin{column}{0.3\textwidth}

\includegraphics[scale=0.3]{aureo}

\end{column}
\end{columns}

\end{frame}
     
\begin{frame}     

\frametitle{Conclusione}
\framesubtitle{\textit{Pur mantenendo le forme del purismo...}}

\includegraphics[height=0.8\paperheight]{6114}

\end{frame}

\begin{frame}     

\frametitle{Conclusione}
\framesubtitle{\textit{Un po' di serenità per Madame Savoye...e non solo! }}
 
\begin{itemize}   
   \item<1-| alert@1> \emph{impianto fotovoltaico} 
   \item<2-| alert@2> \emph{infissi oscuranti} (non solo per risolvere il problema di spifferi e rumore!)
   \item<3-| alert@3> \emph{struttura di raccolta delle acque piovane} ( non solo per risolvere il problema della \emph{fluage}! )
   \item<4-| alert@4> \emph{rivestimento facciata} (non solo per risolvere il problema della muffa!)
    \end{itemize}

\end{frame}    

\begin{frame}
\frametitle{Link}


\small{
\color{blue}{{
\url{http://costruire.laterizio.it/costruire/_pdf/n109/109_68_71.pdf}
\url{http://www.rodoni.ch/corbusier2.pdf} \\ 
\vspace{2mm}
\url{http://orsiniarch.altervista.org/villasavoye/} \\
\vspace{2mm}
\url{http://www.liceogmarconi.gov.it/index.php?option=com_content&view=article&id=760:la-ville-savoye&catid=125:disegno&Itemid=113}\\
\vspace{2mm}
\url{https://it.wikipedia.org/wiki/Le_Corbusier}\\
\vspace{2mm}
\url{http://www.aup.it/wp-content/uploads/2011/10/DARCH2_03_27-OTT.pdf}}}}
\vspace{2mm}
\end{frame}

\end{document}